\documentclass[8pt]{extarticle}
\usepackage{amsfonts}
\usepackage{amsmath}

\begin{document}

\begin{center}
\large{\textbf{Mathematics: Number Systems}}
\end{center}


\begin{tabular}{p{8cm}r}
\textit{Duration: $1\frac{1}{2}$ hours} & \textit{$30$ Points}
\end{tabular}\\

\textsc{ This question paper contains six questions on one page. Please answer all questions. Each question carries 5 points.}\\

\textit{Some instructions and guidelines:\\}
\begin{itemize}
\item A \textit{set} is a collection of items having a specific property. For example, $\mathbb{Q}$ is the set of all rational numbers. 
\item To denote that a number $x$ belongs to a specific set $\mathbb{X}$ we use the notation $x \in \mathbb{X}$, which is read as ``$x$ belongs to or is in the set $\mathbb{X}$''.
\item A set $\mathbb{A}$ is a \textit{subset} of set $\mathbb{B}$ if every element of $\mathbb{A}$ is also an element of $\mathbb{B}$. In addition to this, $\mathbb{B}$  may or may not have any other elements. We write this as: $\mathbb{A} \subseteq \mathbb{B}$.
\item The \textit{cardinality} of a set is the number of elements in it. It is notationally written as $\left\vert{\mathbb{S}}\right\vert$.
\item These are some set notations that might come in handy to you:
\begin{itemize}
\item $\mathbb{R}$ : Set of all real numbers.
\item $\mathbb{Q}$ : Set of all rational numbers.
\item $\mathbb{N}$ : Set of all natural numbers.
\item $\mathbb{Z}$ : Set of all integers.
\item $\mathbb{P}$ : Set of all prime numbers.
\item We may use a superscript $\pm$ to denote a set that contains only positive or negative numbers, respectively.
\end{itemize}
\begin{center}
\textsc{All the best!}
\end{center}
\end{itemize}
\begin{enumerate}
\item Suppose you are given a number $a$. Given that $a<0$ and $a^2 = 2$, justify if $a \in \mathbb{Q^-}$ .
\item Order the following sets as subsets of one another (if it is possible at all, of course - if not, state why). $\mathbb{N}$, $\mathbb{Q^+}$,$\mathbb{R^+}$ $\mathbb{Z^+}$. Can you find two rational numbers between $-1$ and $1$ such that five numbers you have (including the two given) are at exactly the same difference from one another, taken in order? This is also called an \textit{arithmetic progression}, when the difference between any two consecutive numbers in a sequence is the same.
\item Let us take the number $b = 0.9999 \ldots$. Which of the above sets does $b$ belong to? We know that $\pi$ is irrational. That means - we do not the value of $\pi$ very accurately, and we use approximations like $\pi = 3.14159\ldots$. Then, how can we use it if its value keeps varying? Or is this statement wrong? Justify.
\item Solve for $c$ : $\sqrt{c} + \frac{1}{\sqrt{c}} = 4. $ and  $\sqrt{d} - \frac{1}{\sqrt{d}} = 0. $ Let, $e=d^2$ and $f=\sqrt{e}$. Find the possible value(s) of ${ \left( \frac{f}{c} \right)}^{f^{-1}}$.
\item Given that $a^{a^{-1}} = b^{b^{-1}} = c^{c^{-1}}$ and $a^{bc} + b^{ca} + c^{ab} = 729$, show that $a \notin \mathbb{Q}$ but $a \in \mathbb{R}$. What is the value of $a$?
\item Find the value of: $\sqrt{p + \sqrt{p + \sqrt{p + \ldots}}}$, if $p = 2$. If $q = 1 + \sqrt[3]{5} + \sqrt[3]{25}$, find the value of $q^3 - 3q^2 - 12q + 6$.
\end{enumerate}

%\begin{center}\text{End}\end{center}

\end{document}

